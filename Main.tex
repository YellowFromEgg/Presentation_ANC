\documentclass[student, noshadow, lsr, english, aspectratio=169, t]{ITR_LSR_slides}

\setbeamertemplate{frametitle}{%
  \vspace{0.3cm}% Space between top edge and title
  \if@center
    \begin{centering}
      \textbf{\vphantom{Sp}\insertframetitle\vphantom{Sp}}
      \par
    \end{centering}
  \else
    \textbf{\vphantom{Sp}\insertframetitle\vphantom{Sp}}
    \par
  \fi
  \vspace{0.3cm}% Space between title and content
}

% Add top margin to frames without titles
\addtobeamertemplate{frame begin}{%
  \ifx\insertframetitle\@empty
    \vspace{0.8cm}% Space at top of frames without titles
  \fi
}{}

\addbibresource{ref.bib}
\graphicspath{{pics/}{logos/}}

\title{Analysis and Control of Time-Varying and Perturbed Systems}
\presenter{Keno Bürger}
\typeofpres{Advanced Nonlinear Control}

\usepackage{multirow}
\usepackage{graphicx}
\usepackage[T1]{fontenc}
\usepackage{lmodern}
\usepackage{tabularx}
\usepackage{enumitem}
\usepackage{adjustbox}
\usepackage{array}
\usepackage{booktabs}
\usepackage{makecell}

\begin{document}

\begin{frame}
    \titlepage
\end{frame}


%%%%%%%%%%%%%%%%%%%%%%%%%%%%%%%%%%%%%%%%%%%%%%%%%%%%%%
\section{Introduction}

\begin{frame}
	\frametitle{Main Objective}
	\begin{itemize}
		\item Nonlinear Control: Practical Analsysis concerned with exponential stability of time varying systems subject to perturbations (vanishing and non-vanishing)
		\item Nonlinear Systems: Rigorous proof of stability and boundedness theorems/lemmas while being more broadly applicable
		\item More pracitcal theorems for applying Lyapunov theory
	\end{itemize}
\end{frame}

\begin{frame}
	\frametitle{Lyapunov Theory for Time-Varying Systems}
	\begin{itemize}
		\item Definition of Uniform, Asymptotic and exponential stability \cite{muennighoff_s1_2025}
		\item Application of Lyapunov Stability Theorems
	\end{itemize}
\end{frame}

\begin{frame}
	\frametitle{Boundedness and Ultimate Boundedness}
	\begin{itemize}
		\item Differences
		\item Build bridge to non vanishing and vanishing perturbations
	\end{itemize}
\end{frame}

%%%%%%%%%%%%%%%%%%%%%%%%%%%%%%%%%%%%%%%%%%%%%%%%%%%%%
\section{Vanisihing Perturbations}
\begin{frame}
	\frametitle{Perturbation Model for Vanisihing Perturbation Models}
	\begin{itemize}
		\item Exact Modelling rarley feasible due to modelling errors/external disturbances or parameter drift
		\item $\dot{x}=f(x)+g(x,t)$
		\begin{itemize}
			\item f is locally Lipschitz
			\item g is piecewise continuous in t and locally Lipschitz
			\item generally unknown but bounded
		\end{itemize}
		\item $g(0,t)$ and $g(x,t)=0$ for $t\rightarrow\infty$
	\end{itemize}
\end{frame}


\begin{frame}
	\frametitle{Lyapunov Stability Theorems}
	\begin{itemize}
		\item Exponential Stability
		\item Highlight challenges with this approach
	\end{itemize}
\end{frame}

\begin{frame}
	\frametitle{Comparison Functions}
	\begin{itemize}
		\item Differences and Benefits of this approach
		\item Corollary 1
	\end{itemize}
\end{frame}

\begin{frame}
	\frametitle{Exampe: Linear Time-Varying System}
	\begin{itemize}
		\item $\dot{x}=[A(T)+B(t)]x$
		\item Lyapunov function $V(t,x)$ is positive definite and derivative negative definite
		\item $g(t,x) = B(t)x \Rightarrow \|g(t,x)\| \leq \|B(t)\| \cdot \|x\| = \gamma(t) \|x\|$
		\item $\int_{t_0}^{t}\gamma(\tau)d\tau<\epsilon(t-t_o)+\eta$
		\item[$\Rightarrow$] Exponetial stability of nominal system is preserved under vanishing perturbations
	\end{itemize}
\end{frame}

%%%%%%%%%%%%%%%%%%%%%%%%%%%%%%%%%%%%%%%%%%%%%%%%%%%%%
\section{Non-Vanisihing Perturbations}

\begin{frame}
	\frametitle{Perturbation Model for Non-Vanisihing Perturbations}
	\begin{itemize}
		\item Impede the system's convergence towards the origin
		\item Analysing the behavior in terms of boundedness/ultimate boundedness
		\item Gurantee that the state will remain within a small neighborhood around the origin
	\end{itemize}
\end{frame}

\begin{frame}
	\frametitle{Lyapunov Based Conditions for Boundednes}
	\begin{itemize}
		\item Why only boundedness
		\item Lemma 2 for Ultimate Boundedness
	\end{itemize}
\end{frame}

\begin{frame}
	\frametitle{Example: Non-Vanisihing Perturbation in a Nonlinear System}
	\begin{itemize}
		\item As a special case of non-vanishing perturbations
		\item Stability Theorems
		\item Case distinctions
	\end{itemize}
\end{frame}

%%%%%%%%%%%%%%%%%%%%%%%%%%%%%%%%%%%%%%%%%%%%%%%%%%
\section{Summary and Discussion}

\begin{frame}
	\frametitle{Conceptual Links Between Sections}
	\begin{itemize}
		\item As in the report
	\end{itemize}
\end{frame}

\begin{frame}
	\frametitle{Benefits and Drawbacks}
	\begin{itemize}
		\item Benefits and Drawbacks
		\item Final Remarks
	\end{itemize}
\end{frame}


%%%%%%%%%%%%%%%%%%%%%%%%%%%%%%%%%%%%%%%%%%%%%%%%%%%%%
\begin{frame}[allowframebreaks]
    \frametitle{References}
    \nocite{*} 
    \printbibliography[heading=none]
\end{frame}

\end{document}
